% Takes the name of a macro and runs the macro with that name without arguments.
% Make sure that the macro with the given name does not take any arguments.
% For example \runCommandByName{dots} will evaluate to \dots.
\newcommand{\runCommandByName}[1]{\csname #1\endcsname}

% Given a command whose value represents a name, creates a new macro that represents the plural of that name.
% For example, the following:
%   \newcommand{\myFavouriteBread}{toast}
%   \makePluralMacro{myFavouriteBread}
% will create a new macro \toastpl that will evaluate to toasts.
% As an optional argument one may pass an alternative ending replacing "s".
% For instance, the plural of patch should be patches so you should use
%   \makePluralMacro[es]{patch}
\newcommand{\makePluralMacro}[2][s]{\expandafter\newcommand\csname #2pl\endcsname{\runCommandByName{#2}#1}}